\documentclass[11pt,a4paper]{article}
\usepackage[utf8]{inputenc}
\usepackage[T1]{fontenc}
\usepackage{mathptmx} % Times New Roman风格的serif字体
\usepackage[margin=1in]{geometry}
\usepackage{enumitem}
\usepackage{hyperref}
\usepackage{tabularx}
\usepackage{array}
\usepackage{longtable}

% 超链接设置
\hypersetup{
    colorlinks=true,
    linkcolor=blue,
    filecolor=magenta,      
    urlcolor=cyan,
    citecolor=green
}

% 自定义命令
\newcommand{\datecolumn}[1]{\textbf{#1}}
\newcommand{\papertitle}[1]{\textbf{#1}}
\newcommand{\journalname}[1]{\textit{\textbf{#1}}}

% 去除页码
\pagestyle{empty}

\begin{document}

% 标题
\begin{center}
    {\Huge \textbf{Yijie Zheng}} \\[0.5em]
    \href{mailto:zhengyijie23@mails.ucas.ac.cn}{zhengyijie23@mails.ucas.ac.cn}
\end{center}

\vspace{1em}

% 教育经历
\section*{EDUCATION}
\vspace{-0.5em}
\noindent\rule{\textwidth}{0.4pt}
\vspace{0.5em}
\begin{tabularx}{\textwidth}{@{}r@{\hspace{1em}}X@{}}
\datecolumn{2019-2023} & \textbf{Beijing Jiaotong University} \\
& B.S. in Communication Engineering \\
& GPA: 3.95/4 \\[0.5em]

\datecolumn{2023-2026} & \textbf{University of Chinese Academy of Sciences} \\
\datecolumn{(expected)} & M.S. in Remote Sensing \\
& GPA: 3.81/4 \\
\end{tabularx}

\vspace{1em}

% 研究经历
\section*{RESEARCH EXPERIENCE}

\subsection*{Remote Sensing Vision-Language Models}
\papertitle{InstructSAM: A Training-Free Framework for Instruction-Oriented Remote Sensing Object Recognition} 
\href{https://arxiv.org/abs/2505.15818}{[arXiv]} 
\href{https://voyagerxvoyagerx.github.io/InstructSAM}{[Project]} \\
\textbf{Yijie Zheng}, Weijie Wu, Qingyun Li, Xuehui Wang, Xu Zhou, Aiai Ren, Jun Shen, Long Zhao, Guoqing Li, Xue Yang, \journalname{under review}, 2025.
\begin{itemize}[leftmargin=1em]
    \item Proposed InstructSAM, a novel training-free framework that enables object counting, detection, and segmentation in remote sensing imagery through user-defined instructions.
    \item Demonstrated SOTA performance in open-vocabulary, open-ended, and open-subclass settings.
    \item Reduced inference time from linear to near-constant.
\end{itemize}

\vspace{0.5em}

\subsection*{Biodiversity Conservation with Remote Sensing}
\papertitle{Beluga Whale Detection from Satellite Imagery with Point Labels}
\href{https://arxiv.org/abs/2505.12066}{[paper]}
\href{https://github.com/VoyagerXvoyagerx/beluga-seeker}{[code]}
\href{https://www.2025.ieeeigarss.org/view_paper.php?PaperNum=2430&SessionID=1426}{[IGARSS Session]} \\
\textbf{Yijie Zheng}, Jinxuan Yang, Yu Chen, Yaxuan Wang, Yihang Lu, Guoqing Li, in \journalname{Proc. IEEE Int. Geosci. Remote Sens. Symp. (IGARSS)}, 2025.
\begin{itemize}[leftmargin=1em]
    \item Winner of the 4th IEEE GRSS Student Grand Challenge (funded \$6000).
    \item Collaborated with Dr. Cortney Watt from Fisheries and Oceans Canada and supervised by Prof. Adriano Camps, Prof. Paolo Gamba, and Dr. Tianlin Wang from IEEE GRSS.
    \item Developed an automated pipeline for labeling beluga whales and harp seal bounding boxes using point annotations.
\end{itemize}

\vspace{0.5em}

\noindent\papertitle{A Review of Remote Sensing Techniques for Cetacean Monitoring} \\
In preparation, invited to submit to \journalname{IEEE J. Sel. Topics Appl. Earth Observ. Remote Sens.}

\vspace{0.5em}

\subsection*{Ionospheric Remote Sensing}
\papertitle{Segmentation and Edge Detection for Ionogram Automatic Scaling}
\href{https://ieeexplore.ieee.org/document/9955166/}{[paper]}
\href{https://github.com/VoyagerXvoyagerx/Ionogram}{[code]} \\
\textbf{Yijie Zheng}, Xiaoqing Wang, Yefei Luo, Hao Tian, Ziwei Chen, in \journalname{Proc. Int. Conf. Mach. Learn., Cloud Comput. Intell. Min. (MLCCIM)}, 2022.
\begin{itemize}[leftmargin=1em]
    \item Enhanced ionogram structure segmentation with edge detection, significantly improving the accuracy of ionogram scaling.
\end{itemize}

\vspace{0.5em}

\papertitle{A Benchmark for Ionogram Real-Time Object Detection Based on MMYOLO}
\href{https://mmyolo.readthedocs.io/en/dev/recommended_topics/application_examples/ionogram_detection.html}{[doc]}
\href{https://github.com/VoyagerXvoyagerx/Ionogram_scaling}{[code]} \\
\textbf{Yijie Zheng}, \journalname{MMYOLO Application Examples}, 2023
\begin{itemize}[leftmargin=1em]
    \item Formulated ionogram parameter extraction as an object detection problem.
    \item Established a deep learning benchmark for ionospheric structure detection and parameter extraction.
\end{itemize}

\vspace{1em}

% 教学经历
\section*{TEACHING EXPERIENCE}
\begin{itemize}[leftmargin=1em]
    \item \datecolumn{Spring 2023} Teaching Assistant, OpenMMLab AI Camp, Shanghai AI Lab
\end{itemize}

\vspace{1em}

% 竞赛
\section*{COMPETITIONS}
\begin{itemize}[leftmargin=1em]
    \item \datecolumn{2024} \textbf{3rd place} in \href{https://challenge.xfyun.cn/topic/info?type=identification-retrieval}{High-Resolution Remote Sensing Image Content Recognition and Retrieval Challenge} held by iFLYTEK
    \item \datecolumn{2023} \textbf{Winner} in 4th IEEE GRSS Student Grand Challenge
    \item \datecolumn{2022} \textbf{Honorable Mentioned} in Mathematical Contest in Modeling
    \item \datecolumn{2022} \textbf{1st prize} in MathorCup Mathematical Modeling Challenge
    \item \datecolumn{2020} \textbf{1st prize} in National College Students Mathematics Competition
\end{itemize}

\vspace{1em}

% 奖项和荣誉
\section*{AWARDS AND HONORS}
\begin{tabularx}{\textwidth}{@{}r@{\hspace{1em}}X@{}}
\datecolumn{2020, 2022} & \textbf{National Scholarship} \\
& Awarded by the Ministry of Education of the People's Republic of China. Acceptance rate: 1\%. \\[0.5em]

\datecolumn{2023} & \textbf{Highest Honors, BS Communication Engineering, BJTU} \\
& Graduated from Beijing Jiaotong University with a BS with highest honors. \\[0.5em]

\datecolumn{2023} & \textbf{OpenMMLab Active Contributor} \\
& Awarded for the significant contributions to the OpenMMLab ecosystem. \\
\end{tabularx}

\vspace{1em}

% 学术服务
\section*{ACADEMIC SERVICE}
\begin{itemize}[leftmargin=1em]
    \item \datecolumn{2024-2025} Vice Chair, IEEE GRSS Student Branch Chapter (Beijing Section)
    \item \datecolumn{2025} Secretary, \href{https://casa2025.casconf.cn/}{3rd International Training Courses on Space Technology for Global Challenges}
    \item \datecolumn{2025} Member, International Whaling Commission Scientific Committee correspondence group on ``Satellites to Study Whales''
    \item \datecolumn{2023-2024} Chair, Scientific Expedition Association of University of Chinese Academy of Sciences
\end{itemize}

\vspace{1em}

% 语言技能
\section*{LANGUAGE SKILLS}
\begin{itemize}[leftmargin=1em]
    \item Chinese (native), English (working proficiency), Deutsch (learning)
\end{itemize}

\end{document}
